\documentclass[12pt,letterpaper]{article}
\usepackage{graphicx,textcomp}
\usepackage{natbib}
\usepackage{setspace}
\usepackage{fullpage}
\usepackage{color}
\usepackage[reqno]{amsmath}
\usepackage{amsthm}
\usepackage{fancyvrb}
\usepackage{amssymb,enumerate}
\usepackage[all]{xy}
\usepackage{endnotes}
\usepackage{lscape}
\newtheorem{com}{Comment}
\usepackage{float}
\usepackage{hyperref}
\newtheorem{lem} {Lemma}
\newtheorem{prop}{Proposition}
\newtheorem{thm}{Theorem}
\newtheorem{defn}{Definition}
\newtheorem{cor}{Corollary}
\newtheorem{obs}{Observation}
\usepackage[compact]{titlesec}
\usepackage{dcolumn}
\usepackage{tikz}
\usetikzlibrary{arrows}
\usepackage{multirow}
\usepackage{xcolor}
\newcolumntype{.}{D{.}{.}{-1}}
\newcolumntype{d}[1]{D{.}{.}{#1}}
\definecolor{light-gray}{gray}{0.65}
\usepackage{url}
\usepackage{listings}
\usepackage{color}

\definecolor{codegreen}{rgb}{0,0.6,0}
\definecolor{codegray}{rgb}{0.5,0.5,0.5}
\definecolor{codepurple}{rgb}{0.58,0,0.82}
\definecolor{backcolour}{rgb}{0.95,0.95,0.92}

\lstdefinestyle{mystyle}{
	backgroundcolor=\color{backcolour},   
	commentstyle=\color{codegreen},
	keywordstyle=\color{magenta},
	numberstyle=\tiny\color{codegray},
	stringstyle=\color{codepurple},
	basicstyle=\footnotesize,
	breakatwhitespace=false,         
	breaklines=true,                 
	captionpos=b,                    
	keepspaces=true,                 
	numbers=left,                    
	numbersep=5pt,                  
	showspaces=false,                
	showstringspaces=false,
	showtabs=false,                  
	tabsize=2
}
\lstset{style=mystyle}
\newcommand{\Sref}[1]{Section~\ref{#1}}
\newtheorem{hyp}{Hypothesis}

\title{Problem Set 3}
\date{Due: March 26, 2023}
\author{Applied Stats II}


\begin{document}
	\maketitle
	\section*{Instructions}
	\begin{itemize}
	\item Please show your work! You may lose points by simply writing in the answer. If the problem requires you to execute commands in \texttt{R}, please include the code you used to get your answers. Please also include the \texttt{.R} file that contains your code. If you are not sure if work needs to be shown for a particular problem, please ask.
\item Your homework should be submitted electronically on GitHub in \texttt{.pdf} form.
\item This problem set is due before 23:59 on Sunday March 26, 2023. No late assignments will be accepted.
	\end{itemize}

	\vspace{.25cm}
\section*{Question 1}
\vspace{.25cm}
\noindent We are interested in how governments' management of public resources impacts economic prosperity. Our data come from \href{https://www.researchgate.net/profile/Adam_Przeworski/publication/240357392_Classifying_Political_Regimes/links/0deec532194849aefa000000/Classifying-Political-Regimes.pdf}{Alvarez, Cheibub, Limongi, and Przeworski (1996)} and is labelled \texttt{gdpChange.csv} on GitHub. The dataset covers 135 countries observed between 1950 or the year of independence or the first year forwhich data on economic growth are available ("entry year"), and 1990 or the last year for which data on economic growth are available ("exit year"). The unit of analysis is a particular country during a particular year, for a total $>$ 3,500 observations. 

\begin{itemize}
	\item
	Response variable: 
	\begin{itemize}
		\item \texttt{GDPWdiff}: Difference in GDP between year $t$ and $t-1$. Possible categories include: "positive", "negative", or "no change"
	\end{itemize}
	\item
	Explanatory variables: 
	\begin{itemize}
		\item
		\texttt{REG}: 1=Democracy; 0=Non-Democracy
		\item
		\texttt{OIL}: 1=if the average ratio of fuel exports to total exports in 1984-86 exceeded 50\%; 0= otherwise
	\end{itemize}
	
\end{itemize}
\newpage
\noindent Please answer the following questions:

\begin{enumerate}
	\item Construct and interpret an unordered multinomial logit with \texttt{GDPWdiff} as the output and "no change" as the reference category, including the estimated cutoff points and coefficients.
	\begin{itemize}
		\item[Question 1a]
	\lstinputlisting[language=R, firstline=33, lastline=48]{ps3.R}  
	\end{itemize}
	 \begin{lstlisting}[language=R]	
	 	  summary of model
	 	  Call:
	 	  multinom(formula = GDPWdiff ~ REG + OIL, data = data)
	 	  
	 	  Coefficients:
	 	  (Intercept)       REG        OIL
	 	  neutral   -3.8011902 -1.351703 -7.9240683
	 	  positive   0.7284081  0.389905 -0.2076511
	 	  
	 	  Std. Errors:
	 	  (Intercept)        REG        OIL
	 	  neutral   0.27014596 0.75825317 32.9772055
	 	  positive  0.04789662 0.07552484  0.1158094
	 	  
	 	  Residual Deviance: 4678.728 
	 	  AIC: 4690.728 
	 	  
	 	  Exp coeff of Model
	 	           (Intercept)       REG          OIL
	 	  neutral   0.02234416 0.2587991 0.0003619269
	 	  positive  2.07177984 1.4768404 0.8124904479
	 	  
	 	  P-values
	 	           (Intercept)          REG        OIL
	 	  neutral            0 7.464265e-02 0.81010602
	 	  positive           0 2.435359e-07 0.07296613
	\end{lstlisting}
	\begin{itemize}
		\item  The p-values indicate that the estimated coefficients for REG and OIL are statistically significant for predicting the response variable (GDPWdiff) at a 5 percent  significance level.
	
	The results suggest that non-democratic countries are more likely to experience a positive GDPWdiff compared to democratic countries, and countries with a lower ratio of fuel exports to total exports are more likely to experience a positive GDPWdiff compared to countries with a higher ratio.
		\end{itemize}
	\item Construct and interpret an ordered multinomial logit with \texttt{GDPWdiff} as the outcome variable, including the estimated cutoff points and coefficients.
		\begin{itemize}
		\item[Question 1b]
		\lstinputlisting[language=R, firstline=58, lastline=73]{ps3.R}  
		\end{itemize}
	 \begin{lstlisting}[language=R]	
	Summary
	Call:
	polr(formula = GDPWdiff ~ REG + OIL, data = data, Hess = TRUE)
	
	Coefficients:
	Value Std. Error t value
	REG  0.3901    0.07553   5.165
	OIL -0.2080    0.11581  -1.796
	
	Intercepts:
	Value    Std. Error t value 
	negative|no change  -0.7285   0.0479   -15.2097
	no change|positive  -0.7285   0.0479   -15.2091
	
	Residual Deviance: 4482.588 
	AIC: 4490.588 
	(16 observations deleted due to missingness)
	P value
	                        Value Std. Error    t value      p value      p value
	REG                 0.3901136 0.07552803   5.165150 2.402462e-07 2.402462e-07
	OIL                -0.2080302 0.11580601  -1.796368 7.243602e-02 7.243602e-02
	negative|no change -0.7285030 0.04789736 -15.209670 3.050777e-52 3.050777e-52
	no change|positive -0.7284712 0.04789721 -15.209054 3.079625e-52 3.079625e-52
	Odds ratio
	           OR    2.5 %   97.5 %
	REG 1.4771486 1.273332 1.714715
	OIL 0.8121825 0.647811 1.023041
\end{lstlisting}
\begin{itemize}
	\item he results show that for REG, the odds ratio is 1.477 with a 95 percent  confidence interval of 1.273 to 1.715, which means that for every one unit increase in REG, the odds of being in a higher GDPWdiff category (positive or negative) are 1.477 times higher, with a range of 1.273 to 1.715.
	
	For OIL, the odds ratio is 0.812 with a 95 percent confidence interval of 0.648 to 1.023. This means that for every one unit increase in OIL, the odds of being in a higher GDPWdiff category are 0.812 times lower, with a range of 0.648 to 1.023. Note that the confidence interval for OIL includes 1, which suggests that there is not enough evidence to conclude that OIL has a significant effect on GDPWdiff.
\end{itemize}
\end{enumerate}

\section*{Question 2} 
\vspace{.25cm}

\noindent Consider the data set \texttt{MexicoMuniData.csv}, which includes municipal-level information from Mexico. The outcome of interest is the number of times the winning PAN presidential candidate in 2006 (\texttt{PAN.visits.06}) visited a district leading up to the 2009 federal elections, which is a count. Our main predictor of interest is whether the district was highly contested, or whether it was not (the PAN or their opponents have electoral security) in the previous federal elections during 2000 (\texttt{competitive.district}), which is binary (1=close/swing district, 0="safe seat"). We also include \texttt{marginality.06} (a measure of poverty) and \texttt{PAN.governor.06} (a dummy for whether the state has a PAN-affiliated governor) as additional control variables. 

\begin{enumerate}
	\item [(a)]
	Run a Poisson regression because the outcome is a count variable. Is there evidence that PAN presidential candidates visit swing districts more? Provide a test statistic and p-value.
	\begin{itemize}
	\item[lstinputlisting]
	\lstinputlisting[language=R, firstline=78, lastline=89]{ps3.R}  
	 \begin{lstlisting}[language=R]	
	summary
	Call:
	glm(formula = PAN.visits.06 ~ competitive.district + marginality.06 + 
	PAN.governor.06, family = poisson, data = dat)
	
	Deviance Residuals: 
	Min       1Q   Median       3Q      Max  
	-2.2309  -0.3748  -0.1804  -0.0804  15.2669  
	
	Coefficients:
	Estimate Std. Error z value Pr(>|z|)    
	(Intercept)          -3.81023    0.22209 -17.156   <2e-16 ***
	competitive.district -0.08135    0.17069  -0.477   0.6336    
	marginality.06       -2.08014    0.11734 -17.728   <2e-16 ***
	PAN.governor.06      -0.31158    0.16673  -1.869   0.0617 .  
	---
	
	(Dispersion parameter for poisson family taken to be 1)
	
	Null deviance: 1473.87  on 2406  degrees of freedom
	Residual deviance:  991.25  on 2403  degrees of freedom
	AIC: 1299.2
	
	Number of Fisher Scoring iterations: 7
	
	\end{lstlisting}
	\item The output shows that all the independent variables are statistically significant at the 5 percent level, except for "competitive.district". The model also reports the null and residual deviances, and the AIC (Akaike Information Criterion) value, which is used to compare the relative goodness-of-fit of different models.
	\end{itemize}
	\item [(b)]
	Interpret the \texttt{marginality.06} and \texttt{PAN.governor.06} coefficients.
	 \begin{lstlisting}[language=R]	
	 	      Model$coeffecients
	 	         (Intercept) competitive.district       marginality.06      PAN.governor.06 
	 			-3.81023498          -0.08135181          -2.08014361          -0.31157887 
	 	\end{lstlisting}
	\item [(c)]
	Provide the estimated mean number of visits from the winning PAN presidential candidate for a hypothetical district that was competitive (\texttt{competitive.district}=1), had an average poverty level (\texttt{marginality.06} = 0), and a PAN governor (\texttt{PAN.governor.06}=1).
		\begin{itemize}
		\item[lstinputlisting]
		\lstinputlisting[language=R, firstline=95, lastline=99]{ps3.R}  
		\end{itemize}
		 \begin{lstlisting}[language=R]	
	      1 
	1.01506 
	\end{lstlisting}
\end{enumerate}

\end{document}
