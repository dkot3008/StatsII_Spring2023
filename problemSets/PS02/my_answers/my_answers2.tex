\documentclass[12pt,letterpaper]{article}
\usepackage{graphicx,textcomp}
\usepackage{natbib}
\usepackage{setspace}
\usepackage{fullpage}
\usepackage{color}
\usepackage[reqno]{amsmath}
\usepackage{amsthm}
\usepackage{fancyvrb}
\usepackage{amssymb,enumerate}
\usepackage[all]{xy}
\usepackage{endnotes}
\usepackage{lscape}
\newtheorem{com}{Comment}
\usepackage{float}
\usepackage{hyperref}
\newtheorem{lem} {Lemma}
\newtheorem{prop}{Proposition}
\newtheorem{thm}{Theorem}
\newtheorem{defn}{Definition}
\newtheorem{cor}{Corollary}
\newtheorem{obs}{Observation}
\usepackage[compact]{titlesec}
\usepackage{dcolumn}
\usepackage{tikz}
\usetikzlibrary{arrows}
\usepackage{multirow}
\usepackage{xcolor}
\newcolumntype{.}{D{.}{.}{-1}}
\newcolumntype{d}[1]{D{.}{.}{#1}}
\definecolor{light-gray}{gray}{0.65}
\usepackage{url}
\usepackage{listings}
\usepackage{color}

\definecolor{codegreen}{rgb}{0,0.6,0}
\definecolor{codegray}{rgb}{0.5,0.5,0.5}
\definecolor{codepurple}{rgb}{0.58,0,0.82}
\definecolor{backcolour}{rgb}{0.95,0.95,0.92}

\lstdefinestyle{mystyle}{
	backgroundcolor=\color{backcolour},   
	commentstyle=\color{codegreen},
	keywordstyle=\color{magenta},
	numberstyle=\tiny\color{codegray},
	stringstyle=\color{codepurple},
	basicstyle=\footnotesize,
	breakatwhitespace=false,         
	breaklines=true,                 
	captionpos=b,                    
	keepspaces=true,                 
	numbers=left,                    
	numbersep=5pt,                  
	showspaces=false,                
	showstringspaces=false,
	showtabs=false,                  
	tabsize=2
}
\lstset{style=mystyle}
\newcommand{\Sref}[1]{Section~\ref{#1}}
\newtheorem{hyp}{Hypothesis}

\title{Problem Set 3}
\date{Due: March 26, 2023}
\author{Applied Stats II}


\begin{document}
	\maketitle
	\section*{Instructions}
	\begin{itemize}
		\item Please show your work! You may lose points by simply writing in the answer. If the problem requires you to execute commands in \texttt{R}, please include the code you used to get your answers. Please also include the \texttt{.R} file that contains your code. If you are not sure if work needs to be shown for a particular problem, please ask.
		\item Your homework should be submitted electronically on GitHub in \texttt{.pdf} form.
		\item This problem set is due before 23:59 on Sunday March 26, 2023. No late assignments will be accepted.
	%	\item Total available points for this homework is 80.
	\end{itemize}

	
	%	\vspace{.25cm}
	
%\noindent In this problem set, you will run several regressions and create an add variable plot (see the lecture slides) in \texttt{R} using the \texttt{incumbents\_subset.csv} dataset. Include all of your code.

	\vspace{.25cm}
%\section*{Question 1} %(20 points)}
%\vspace{.25cm}
\noindent Question 1
We are interested in how governments’ management of public resources impacts economic prosperity. Our data come from Alvarez, Cheibub, Limongi, and Przeworski (1996) and is labelled gdpChange.csv on GitHub. The dataset covers 135 countries observed between 1950 or the year of independence or the first year forwhich data on economic growth are available (”entry year”), and 1990 or the last year for which data on economic growth are available (”exit year”). The unit of analysis is a particular country during a particular year, for a total > 3,500 observations.
• Response variable:
– GDPWdiff: Difference in GDP between year t and t−1. Possible categories include:
”positive”, ”negative”, or ”no change” • Explanatory variables:
– REG: 1=Democracy; 0=Non-Democracy
– OIL: 1=if the average ratio of fuel exports to total exports in 1984-86 exceeded 50%; 0= otherwise
1
Please answer the following questions:
1. Construct and interpret an unordered multinomial logit with GDPWdiff as the output and ”no change” as the reference category, including the estimated cutoff points and coefficients.
2. Construct and interpret an ordered multinomial logit with GDPWdiff as the outcome variable, including the estimated cutoff points and coefficients. \\

\noindent Load in the data labeled \texttt{climateSupport.csv} on GitHub, which contains an observational study of 8,500 observations.

\begin{itemize}
	\item
	Response variable: 
	\begin{itemize}
		\item \texttt{choice}: 1 if the individual agreed with the policy; 0 if the individual did not support the policy
	\end{itemize}
	\item
	Explanatory variables: 
	\begin{itemize}
		\item
		\texttt{countries}: Number of participating countries [20 of 192; 80 of 192; 160 of 192]
		\item
		\texttt{sanctions}: Sanctions for missing emission reduction targets [None, 5\%, 15\%, and 20\% of the monthly household costs given 2\% GDP growth]
		
	\end{itemize}
	
\end{itemize}

\newpage
\noindent Please answer the following questions:

\begin{enumerate}
	\item
	Remember, we are interested in predicting the likelihood of an individual supporting a policy based on the number of countries participating and the possible sanctions for non-compliance.
	\begin{enumerate}
		\item [] Fit an additive model. Provide the summary output, the global null hypothesis, and $p$-value. Please describe the results and provide a conclusion.
		%\item
		%How many iterations did it take to find the maximum likelihood estimates?
	\end{enumerate}
\lstinputlisting[language=R, firstline=38, lastline=70]{my_answers2.R}  
	\begin{enumerate}
		\item [Output 1] The coefficients of the model represent the change in the log odds of supporting the policy associated with a one-unit change in the independent variable. For example, a one-unit increase in "countries" is associated with a 0.458452 increase in the log odds of supporting the policy, holding all other variables constant.
		\item[Output 2] The output of the ANODEV test shows that the p-value is much smaller than 0.05, which indicates that the addition of the explanatory variables significantly improves the model fit. Therefore, we can reject the null hypothesis that the intercept-only model is adequate, and we can conclude that the model including the number of participating countries and sanctions is a better fit for the data.
		\item[Output 3] Confidence intervals
	\end{enumerate}
\begin{lstlisting}[language=R]
Output 1
Call:
glm(formula = choice ~ ., family = "binomial", data = data)

Deviance Residuals: 
Min       1Q   Median       3Q      Max  
-1.4259  -1.1480  -0.9444   1.1505   1.4298  

Coefficients:
Estimate Std. Error z value Pr(>|z|)    
(Intercept) -0.005665   0.021971  -0.258 0.796517    
countries.L  0.458452   0.038101  12.033  < 2e-16 ***
countries.Q -0.009950   0.038056  -0.261 0.793741    
sanctions.L -0.276332   0.043925  -6.291 3.15e-10 ***
sanctions.Q -0.181086   0.043963  -4.119 3.80e-05 ***
sanctions.C  0.150207   0.043992   3.414 0.000639 ***
---
stars not wokring so replaced by
Signif. codes:  0 *** 0.001 ** 0.01 * 0.05 . 0.1 , 1

(Dispersion parameter for binomial family taken to be 1)

Null deviance: 11783  on 8499  degrees of freedom
Residual deviance: 11568  on 8494  degrees of freedom
AIC: 11580

Number of Fisher Scoring iterations: 4
#######################################
Output 2
	Analysis of Deviance Table
	
	Model 1: choice ~ 1
	Model 2: choice ~ countries + sanctions
	Resid. Df Resid. Dev Df Deviance  Pr(>Chi)    
	1      8499      11783                          
	2      8494      11568  5   215.15 < 2.2e-16 ***
	---
Signif. codes:  0 *** 0.001 ** 0.01 * 0.05 . 0.1 , 1
#######################################
Output 3
Confint
                2.5 %    97.5 %
(Intercept) 0.9524387 1.0381058
countries.L 1.4679656 1.7044456
countries.Q 0.9189295 1.0667733
sanctions.L 0.6959419 0.8267142
sanctions.Q 0.7654570 0.9094241
sanctions.C 1.0661324 1.2667989
\end{lstlisting}
\begin{figure}[H]
	\centering
	\includegraphics[width=0.7\linewidth]{"Output 4"}
	\caption{confidence intervals}
	\label{fig:output-4}
\end{figure}



	\item
	If any of the explanatory variables are significant in this model, then:
	\begin{enumerate}
		\item
		For the policy in which nearly all countries participate [160 of 192], how does increasing sanctions from 5\% to 15\% change the odds that an individual will support the policy? (Interpretation of a coefficient)
%		\item
%		For the policy in which very few countries participate [20 of 192], how does increasing sanctions from 5\% to 15\% change the odds that an individual will support the policy? (Interpretation of a coefficient)

\lstinputlisting[language=R, firstline=72, lastline=83]{my_answers2.R}  
\begin{lstlisting}[language=R]
Increasing sanctions from 5% to 15% increases the odds of supporting the policy by a factor of 0.98 
\end{lstlisting}
	
		\item
		What is the estimated probability that an individual will support a policy if there are 80 of 192 countries participating with no sanctions? 
\lstinputlisting[language=R, firstline=87, lastline=93]{my_answers2.R}  
\begin{lstlisting}[language=R]

Call:
glm(formula = choice ~ sanctions, family = binomial, data = policy_80)

Deviance Residuals: 
Min      1Q  Median      3Q     Max  
-1.315  -1.157  -1.033   1.135   1.329  

Coefficients:
Estimate Std. Error z value Pr(>|z|)    
(Intercept)  0.005309   0.038188   0.139 0.889433    
sanctions.L -0.383502   0.074974  -5.115 3.13e-07 ***
sanctions.Q -0.258570   0.076377  -3.385 0.000711 ***
sanctions.C  0.144713   0.077755   1.861 0.062724 .  
---
Signif. codes:  0 *** 0.001 ** 0.01 * 0.05 . 0.1 , 1

(Dispersion parameter for binomial family taken to be 1)

Null deviance: 3874.6  on 2794  degrees of freedom
Residual deviance: 3834.4  on 2791  degrees of freedom
AIC: 3842.4

Number of Fisher Scoring iterations: 4

\end{lstlisting}
		\item
		Would the answers to 2a and 2b potentially change if we included the interaction term in this model? Why? 
		\begin{itemize}
			\item Perform a test to see if including an interaction is appropriate.
			\lstinputlisting[language=R, firstline=96, lastline=99]{my_answers2.R}  
			\begin{lstlisting}[language=R]
			Analysis of Deviance Table
			
			Model 1: choice ~ countries + sanctions
			Model 2: choice ~ countries * sanctions
			Resid. Df Resid. Dev Df Deviance Pr(>Chi)
			1      8494      11568                     
			2      8488      11562  6   6.2928   0.3912
			\end{lstlisting}
		\item The output of the ANODEV test shows that the p-value is 0.3912, which is greater than the usual significance level of 0.05. This suggests that including the interaction term does not significantly improve the model fit, and the additive model is adequate. Therefore, we can conclude that the relationship between the response variable and the explanatory variables is well-captured by the additive model that includes only the main effects.
		
		\end{itemize}
	\end{enumerate}
	\end{enumerate}


\end{document}
